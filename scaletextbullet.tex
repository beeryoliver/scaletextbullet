\documentclass{beery}

\NewDocumentCommand\textbulletfactor{}{\cs{textbullet} factor}


\begin{document}


\section*
  {%
    \SetFontStep*{1}%
    The \pkg{scaletextbullet} package%
  }

Resize the \cs{textbullet} without changing its vertical center

Oliver Beery

Version 1.1.1\quad{}6 February 2025


\section{Introduction}
\label{sec:intro}

\subsection{About}

The \pkg{scaletextbullet} package enables the user to resize the \cs{textbullet} without moving its vertical center, unlike direct usage of the \LaTeXe{} and \pkg{expl3} commands \cs{scalebox} and \cs{box_scale:}.
This process is not fully automated\---the user must use \cs{SetTextBulletFactor} to set the \textbulletfactor{} to the correct value to display the resized \cs{textbullet} at the correct height.
The \textbulletfactor{} is the ratio of the width of the \cs{textbullet}, excluding its empty space, to its width, including its empty space.
One way of estimating the \textbulletfactor{} is by using \cs{scaletextbulletdebug}.

This package provides a solution that works only in text mode.
For a solution that works only in math mode, see the linked \TeX{} Stack Exchange thread.%
\footnote{\url{https://tex.stackexchange.com/questions/119319/how-to-correctly-shrink-the-bullets-of-itemize}}

\subsection{Loading the package}

\listheading{Requirements:}
\begin{itemize}
  \item \LaTeXe{} version 2023-11-01 or newer
  \item \pkg{l3kernel} version 2023-10-10 or newer
\end{itemize}

You may need to ensure that your \LaTeX{} installation is up\-/to\-/date before using this package.


\section{Commands}
\label{sec:commands}

This package defines some commands whose argument takes a \meta{floating point expression} or \meta{integer expression}.
This syntax has the same representation as the arguments to \cs{fpeval} and \cs{inteval}, documented in \pkg{usrguide}.

\begin{mydisplaycode}
  \cs{SetTextBulletFactor} \marg{floating point expression}
\end{mydisplaycode}

Sets the \textbulletfactor{} to the result of computing the \meta{floating point expression}.
The \textbulletfactor{} is the ratio of the width of the \cs{textbullet}, excluding its empty space, to its width, including its empty space.
The scope of the effect is local to the current group.
The initial \textbulletfactor{} is \num{0.4}\---this matches the dimensions of the \cs{textbullet} of the Latin Modern font at size \qty{10}{pt}.
\pagebreak%manual

\begin{mydisplaycode}
  \cs{ScaleTextBullet} \marg{floating point expression}
\end{mydisplaycode}

Prints a \cs{textbullet} with its size scaled by the result of computing the \meta{floating point expression}.
The new \cs{textbullet} will be printed with the same vertical center only if the \textbulletfactor{} is set to the correct value.
Cannot be used in math mode.

\begin{mydisplaycode}
  \cs{ScaleTextBullets} \oarg{floating point expression} \marg{integer expression}
\end{mydisplaycode}

Prints a number of \cs{textbullet}s equal to the value of \meta{integer expression} with about the same total area as the original \cs{textbullet}.%
\footnote
  {%
    In calculating the total area, I have approximated each \cs{textbullet} as a perfect circle, but, of course, the actual shape depends on the font used.%
  }
If the optional argument is used, the size of each \cs{textbullet} is instead scaled by the result of computing the \meta{floating point expression}.
The new \cs{textbullet} will be printed with the same vertical center only if the \textbulletfactor{} is set to the correct value.
Cannot be used in math mode.

\begin{mydisplaycode}
  \cs{scaletextbulletdebug}
\end{mydisplaycode}

This command is provided only to help the user estimate the \textbulletfactor{}.
Prints \num{15} consecutive \cs{textbullet}s with decreasing sizes.
The \cs{textbullet}s are followed by the original \cs{textbullet} inside a framed box.
The framed box has width equal to the \textbulletfactor{} \texttimes{} the total width of the \cs{textbullet} (this includes its empty space).
The \textbulletfactor{} is set to the correct value when the \num{15} consecutive \cs{textbullet}s have the same vertical center and the \cs{textbullet} fits nicely inside the framed box.
Cannot be used in math mode.


\section{Application}
\label{sec:application}

I wrote this package primarily to create nicer\-/looking itemized lists.
The default list labels in \LaTeX{} (and other programs) fail to communicate the list level within the list hierarchy:

\begin{multicols}{2}
  \begin{itemize}[label=\textbullet]
    \item List level 1
    \begin{itemize}[label=\textbf{--}]
      \item List level 2
      \item List level 2
      \begin{itemize}[label=\textasteriskcentered]
        \item List level 3
      \end{itemize}
    \end{itemize}
    \item List level 1
    \begin{itemize}[label=\textbf{--}]
      \item List level 2
      \begin{itemize}[label=\textasteriskcentered]
        \item List level 3
        \item List level 3
      \end{itemize}
    \end{itemize}
  \end{itemize}
\end{multicols}

This contrasts with traditional enumerated list structures where the list level is obvious from the numbering of the list label:

\begin{multicols}{2}
  \begin{enumerate}
    \item List level 1
    \begin{enumerate}
      \item List level 2
      \item List level 2
      \begin{enumerate}
        \item List level 3
      \end{enumerate}
    \end{enumerate}
    \item List level 1
    \begin{enumerate}
      \item List level 2
      \begin{enumerate}
        \item List level 3
        \item List level 3
      \end{enumerate}
    \end{enumerate}
  \end{enumerate}
\end{multicols}

This package allows the user to create nice\-/looking itemized lists using \cs{ScaleTextBullets}:

\begin{multicols}{2}
  \begin{itemize}
    \item List level 1
    \begin{itemize}
      \item List level 2
      \item List level 2
      \begin{itemize}
        \item List level 3
      \end{itemize}
    \end{itemize}
    \item List level 1
    \begin{itemize}
      \item List level 2
      \begin{itemize}
        \item List level 3
        \item List level 3
      \end{itemize}
    \end{itemize}
  \end{itemize}
\end{multicols}

The visual effect may be more clear with different fonts.
This example uses STIX Two Text and Source Serif 4, respectively.

\begingroup
  \SetTextBulletFactor{0.54}
  \newfontfamilystixtwo
  \begin{multicols}{2}
    \begin{itemize}
      \item List level 1
      \begin{itemize}
        \item List level 2
        \item List level 2
        \begin{itemize}
          \item List level 3
        \end{itemize}
      \end{itemize}
      \item List level 1
      \begin{itemize}
        \item List level 2
        \begin{itemize}
          \item List level 3
          \item List level 3
        \end{itemize}
      \end{itemize}
    \end{itemize}
  \end{multicols}
\endgroup

\begingroup
  \SetTextBulletFactor{0.75}
  \newfontfamilysourceseriffour
  \begin{multicols}{2}
    \begin{itemize}
      \item List level 1
      \begin{itemize}
        \item List level 2
        \item List level 2
        \begin{itemize}
          \item List level 3
        \end{itemize}
      \end{itemize}
      \item List level 1
      \begin{itemize}
        \item List level 2
        \begin{itemize}
          \item List level 3
          \item List level 3
        \end{itemize}
      \end{itemize}
    \end{itemize}
  \end{multicols}
\endgroup


\section{Implementation notes}
\label{sec:implementation}

The procedure of resizing the \cs{textbullet} without changing its vertical center, including the definition of the \textbulletfactor{}, makes an important assumption:
That the \cs{textbullet} is a perfect circle.
Of course, this is not completely accurate and the actual shape depends on the font used.
This means that the \textbulletfactor{} may not be exactly the ratio of the width of the \cs{textbullet}, excluding its empty space, to its width, including its empty space.

In writing this package, I have referenced a comment on the \TeX{} Stack Exchange by the user egreg.%
\footnote{\url{https://tex.stackexchange.com/questions/620507/how-to-resize-textbullet-without-the-bullet-moving-down/638599\#638599}}
This package uses the same procedure for resizing the \cs{textbullet} without changing its vertical center.


\section{Programming}
\label{sec:programming}

This package is written in \pkg{expl3}, but does not provide any public functions or variables.


\end{document}